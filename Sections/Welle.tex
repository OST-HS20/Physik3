\section{Welle}
Eine Welle ist eine Störung eines Gleichgewichtszustandes, die sich im Raum $x,y,z$ Koordinaten ausbreitet. Bei Transversalwellen ist die Störung quer zur Ausbreitungsrichtung, bei Longitudinalwellen parallel. Bei elektromagnetischen Wellen (zB Licht) gibt es ausschliesslich Transversalwellen, Schall hingegen nur Longitudinalwellen. Es wird\underline{ Energie transportiert, aber kein Material}. Die allgemeine Wellengleichung ist gegeben durch Amlitude $\zeta_o$, der Wellenlänge $k = \frac{2\pi}{\lambda}$ und mittels Wellengeschwindigkeit $u$ ergibt sich $\omega = k \cdot u$, $\varphi$ kann durch einsetzen bestimmt werden \kuchling{231}:
                  
\[
\zeta(x, t) = \zeta_o\sin(\omega t - kx)
\]

\begin{center}
	\includegraphics[width=0.8\columnwidth]{Images/wellen}
\end{center}                                                 

\subsection{Harmonische Welle}
\includegraphics[width=\columnwidth]{Images/harm.welle}
\begin{formulaexpanded}
	{k = \frac{\omega}{u} = \frac{2\pi}{\lambda} \quad \omega = 2\pi f = \frac{2\pi}{T}}
	k & Faktor & [] \\
	u & Lichtgeschw. & [m/s] \\
	\lambda & Wellenlänge & [m/s] \\
	\omega & Kreisgeschw. & [m/s] \\
	T & Periodendauer & [s]\\
	f & Frequenz & [Hz/s^{-1}]
\end{formulaexpanded}   

\subsection{Wellengeschwindigkeit}\label{wgeschw}
\textbf{Festkörper} (Longitudinal oder Transversal):
\begin{formulaexpanded}
	{u_1 = \sqrt{\frac{E}{\rho}} \qquad u_2 = \sqrt{\frac{G}{\rho}} \qquad u_3 =\sqrt{\frac{F}{\rho A}} }
	u_1 & Elastische Longitudenwelle & [m/s] \\
	u_2 & Elastische Transversalwelle & [m/s] \\
	u_3 & Transversalwelle auf einem Seil & [m/s] \\
	E & Elastizitätsmodule & [] \\
	G & Schubmodul & [] \\
	\rho & Dichte & [] \\
	A & Querschnitt & [m] \\
\end{formulaexpanded}

\textbf{Fluide} (nur Longitudinal):
\begin{formulaexpanded}
	{u_1 = \sqrt{\frac{1}{\rho\kappa}} \qquad u_2 = \sqrt{\frac{\varkappa p}{\rho}} = \sqrt{\frac{\varkappa RT}{M}} \qquad u_3 = \frac{c}{n}}
	u_1 & Schallwelle in Fluid & [m/s] \\
	u_2 & Schallwelle in Gasen & [m/s] \\
	u_3 & elektromagnetische Wellen & [m/s] \\
	\varkappa & Adiabatenexponent & [] \\
	p & Druck & [Pa] \\
	\kappa & Kompressibilität & [] \\
	c & Lichtgeschwindigkeit & [m/s] \\
	n & Brechungsindex & [] \\
\end{formulaexpanded}                                                

\noindent\textbf{Hinweis:} Die Mittlere Molmasse der Luft ist $0.02883\frac{kg}{mol}$ und Adiabatenexponent der Luft ist $1.4$ mit Gaskonstante $R=8.314$

\subsection{Dopplereffekt}\kuchling{342}
\subsubsection{Akustisch}
Für bewegter Beobachter und bewegte Quelle:\\
\includegraphics[width=0.4\columnwidth]{Images/dopplereffekt}

\begin{formula}
	{f_B = \frac{u + v_B\cos(\alpha_B)}{u-v_Q\cos(\alpha_Q)}f_Q}
	x_B & Beobachter & [] \\
	x_Q & Quelle & [] \\
	u & Schallgeschw. & [m/s]
\end{formula}    
~\\ ~\\
Für bewegter Quelle mit Winkel. Ohne Winkel $\cos(0) = 1$. Wenn Distanz von Q und B kleiner wird $-$, bei grösserwerdender Distanz $+$:\\
\includegraphics[width=0.4\columnwidth]{Images/dopplereffekt1}
\begin{formulaexpanded}
	{f_B = \frac{1}{1\mp\frac{v_Q}{u}\cos(\alpha_Q)}f_Q}
	x_B & Beobachter & [] \\
	x_Q & Quelle & [] \\
	u & Schallgeschw. & [m/s]
\end{formulaexpanded}    


\subsubsection{Optisch}
Es spielt nur die relative Bewegung von Beobachter und Quelle eine Rolle\\
\begin{formulaexpanded}
	{f_B = \frac{\sqrt{1-\beta^2}}{1-\beta\cos(\alpha)}f_Q \qquad \beta = \frac{v}{c}} 
	x_B & Beobachter & [] \\
	x_Q & Quelle & [] \\
	u & Schallgeschw. & [m/s] \\
	c & Lichtgeschw. & [m/s]
\end{formulaexpanded}    
Siehe auch Mach'sche Kegel \kuchling{344} für Überschallgeschwindigkeit.

\subsection{Schwebung}\label{schwebung}
Um zB eine Frequenz in GHz zu messen, wird die Schwebung $f_s$ mithilfe einer Amplitudenmodulation $y$ gebildet. Diese Schwebung kann einfacher gemessen und ausgewertet werden.
\begin{center}
	\includegraphics[width=\columnwidth]{Images/schwebung}
\end{center}

\begin{align*}
	y &= A[\sin(2\pi f_1 t) + \sin(2\pi f_2 t)] \\
	&= 2A\cdot\sin\left(\underbrace{\frac{\omega_1 + \omega_2}{2}}_{\overline{\omega}}t\right)\cdot \cos\left(\underbrace{\frac{\omega_1 - \omega_2}{2}}_{\hat{\omega}}t\right) \\
	\\
	f_s &= f_1 - f_2 = \frac{\omega_1 - \omega_2}{2\pi}
\end{align*}

\subsection{Wellenimpdanz}
\noindent Die \textbf{Wellenimpedanz} $Z$ ist gegeben durch
\begin{formulaexpanded}
	{Z = \rho u = \frac{\Delta p_0}{v_0}}
	\Delta p_0 & Druckamplitude & [] \\
	\rho & Dichte des Mediums & [\frac{kg}{m^3}] \\
	u & Geschw. im Medium & [\frac{m}{s}] \\
	v_0 & Geschw.amplitude & [\frac{m}{s}]
\end{formulaexpanded}

\textbf{Schalldruck} \kuchling{349}
\begin{formulaexpanded}
	{\tilde{p}(x,t) = \Delta p_0\cos(\omega t - kx) \qquad \Delta p_0 = \rho u \overbrace{\omega \varepsilon_0}^{v_0} = \rho u v_0}
	\tilde{p}& Schalldruck & [Pa] \\
	\Delta p_0 & Druckamplitude & [] \\
	k & Wellenlänge & [] \\
	\rho & Dichte des Mediums & [\frac{kg}{m^3}] \\
	u & Geschw. im Medium & [\frac{m}{s}] \\
	\omega & Kreisfrequenz & [] \\
	\varepsilon_0 & Schwingungsamplitude & [] \\
	v_0 & Geschw.amplitude & [\frac{m}{s}]
\end{formulaexpanded}

\noindent\textbf{Schallschnelle} $v$ ist die Geschwindigkeit des Punktes in einer Longitudinalwelle in x-Achse.
\begin{formula}
	{v = v_0\cos(\omega t - kx)}
\end{formula}\\

\noindent\textbf{Schallintensität} $I$ verteilt sich gleichmässig im Raum. Dh bei einer Quelle in der Mitte einer Kugel verteilt sich die Intensität proportional zur Oberfläche.
\begin{formula}
	{I = \frac{1}{2}\rho v_0^2 u = \frac{\Delta p_0^2}{2Z}}
\end{formula}\\

\noindent\textbf{Hinweis:} Elektromagnetische Wellen im Vakuum haben einen Widerstand von $Z_0 = 377 \ohm$. Die Wellenimpedanz von elekt.Wellen können mit $Z = Z_0\frac{c}{n}$ berechnet werden, wobei $n$ der Brechungsindex ist.\\

\noindent\textbf{Schallpegel:}
\begin{formulaexpanded}
	{L = 10\log_{10}\left({\frac{I}{I_0}}\right)}
	L & Schallintensitätspegel & [dB] \\
	I_0 = 10^{-12}& Bezugsintensität &  [W/m^2]\\	
\end{formulaexpanded} 

\subsection{Wellenüberlagerung}
Wellen können an einem festen oder losen Ende reflektiert werden. Dabei wird bei einem festen Ende die Welle um $180^\circ$ (Destruktiv) Phasenverschoben, sonst $0^\circ$ (Konstruktive). Wenn die Wellen sich überlagern, insteht \textbf{Inteferenz}. Konstruktive Inteferenz tritt auf, wenn zwei Wellenberge sich Maximieren und so zB die Tonlautstärke verdoppeln. Bei destruktuve Inteferenz ist eine Welle um $\frac{\lambda}{2}$ verschoben und sie löschen sich aus aka Noise-Cancelling

\begin{center}
	\includegraphics[width=0.4\columnwidth]{Images/wellen_überlagern}
\end{center}

Reflexionskoeffizient:
\begin{formula}
	{R = \left(\frac{Z_1-Z_2}{Z_1+Z_2}\right)^2} 
	Z = \rho \cdot u & Wellenimpendanz & \\
	\rho & Dichte & \\
	u & Schallgeschw. &
\end{formula}    

Transmissionskoeffizient:
\begin{formula}
	{T = \frac{4Z_1Z_2}{(Z_1+Z_2)^2}} 
	Z = \rho \cdot u & Wellenimpendanz & \\
	\rho & Dichte & \\
	u & Schallgeschw. &
\end{formula}  

\subsection{Stehendewelle}
Zwei Wellen die gleichzeitig in entgegengesetzter Richtung durch das gleich Medium laufen, überlagern sich zu einer stehenden Welle, wenn sie ein vielfaches der \underline{Eigenfrequenz} sind.\kuchling{233}
\begin{formulaexpanded}
	{f_n = n\cdot\frac{u}{2L} \qquad \lambda_n = \frac{2L}{n}} 
	f_n & Frequenz & \\
	n & Anz. Harmonische Freq. & \\
	u & Wellengeschwindigkeit (Siehe Kap. \ref{wgeschw})& \\
	L & Länge & \\
	\lambda & Wellenlänge & [] \\
\end{formulaexpanded} 
Die Anzahl der harmonischen Frequenzen $n$ kann die Anzahl der Schwingungsknoten $k = n -1$ ermittelen.
\begin{center}
	\includegraphics[width=0.9\columnwidth]{Images/stehendewelle}
\end{center}

\subsection{Beugung}
An der Kante eines engen Spalts bildet sich eine \kuchling{394}
\begin{formulaexpanded}
	{\sin(\alpha) = \frac{\lambda}{b}n \qquad \tan(\alpha) = \frac{a}{d}} 
	b & Spaltbreite & [] \\
	\lambda & Wellenlänge & [] \\
	\varphi & Beugungswinkel & [] \\
	n & n-te Wiederholung & []	
\end{formulaexpanded} 

\begin{center}
	\includegraphics[width=0.6\columnwidth]{Images/beugung}
\end{center}

\subsection{Nachhall}
Wenn eine Schallquelle plötzlich verschwindet, wird ein Nachhall produziert, welcher vom Ohr wahrgenommen wird. Die Dauer des Nachhalles $T_N$ kann folgendermassen berechnet werden:
\begin{formulaexpanded}
	{T_N = 0.16\frac{V}{\sum_{i}\alpha_i A_i}}
	V & Raumvolumen & [m^3] \\
	\alpha & Absorptionsgrad & \\	
	A & Absorbierende Fläche & m^2
\end{formulaexpanded} 
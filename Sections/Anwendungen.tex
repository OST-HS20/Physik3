\section{Anwendungen}
\subsection{Parabolspiegel}
Wobei $f$ die Brennweite ist. Damit lässt sich eine Funktion für den Parabolspiegel finden bei dem der Brennpunkt bestimmt ist.
\[y(x) = \frac{1}{4f}x^2  - f\]

\begin{center}
	\includegraphics[width=0.8\columnwidth]{Images/parabolspiegel}
\end{center}


\subsection{Spiegelungen}
Bei einem Spiegel kann das virtuelle Bild durch eine Spiegelung des Bildpunkts und dem Strahlensatz gefunden werden.
\includegraphics[width=\linewidth]{Images/spiegel_anwendung}


\subsubsection{Optiker}
\begin{formula}
	{D = \frac{1}{f}}
	D & Dioptrie & \\
	f & Brennweite & [m]
\end{formula}

\begin{formula}
	{D = \left(\frac{n_2}{n_1} -1\right)\left(\frac{1}{r_1}+\frac{1}{r_2}\right)}
	D & Dioptrie & \\
	n & Brechungsindex & [] \\
	r_x & Radius & []
\end{formula}
\includegraphics[width=0.5\columnwidth]{Images/optiker}


\subsection{Fotoapperat}
\todo{Woche 4, Freitag 15.10.2021}

\begin{formulaexpanded}
	{H = \left(\frac{d}{f}\right)^2 = q^2 \qquad Z = \frac{1}{q}}
	Z & Blendenzahl & [-] \\
	H & Lichtstärke & [] \\
	f & Brennweite & [] \\
	d &  & []
\end{formulaexpanded}

\subsection{Schwingungen}
\todo{Beispiel mit Feder Prüfung 2019 Aufgabe 2}

\subsection{Pendel}
\todo{Woche 5}
\todo{Schwingdauer \[T(a) = 2\pi\sqrt{\frac{J_S + ma^2}{mga}}\]}

\subsection{Radar}
\todo{Woche 9 Freitag}
Mithilfe einer Amplitudenmodulation von $f$ und $f'$ wird die Schwebungsfrequenz bestimmt, welche grössenordnung $100Hz$ hat, welche ohne probleme gemessen werden kann.
\includegraphics[width=\columnwidth]{Images/radar}


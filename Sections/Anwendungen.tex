\section{Anwendungen}
\subsection{Parabolspiegel}
Wobei $f$ die Brennweite ist. Damit lässt sich eine Funktion für den Parabolspiegel finden bei dem der Brennpunkt bestimmt ist.
\[y(x) = \frac{1}{4f}x^2  - f\]

\begin{center}
	\includegraphics[width=0.8\columnwidth]{Images/parabolspiegel}
\end{center}


\subsection{Spiegelungen}
Bei einem Spiegel kann das virtuelle Bild durch eine Spiegelung des Bildpunkts und dem Strahlensatz gefunden werden.
\includegraphics[width=\linewidth]{Images/spiegel_anwendung}


\subsubsection{Optiker}
\begin{formula}
	{D = \frac{1}{f}}
	D & Dioptrie & \\
	f & Brennweite & [m]
\end{formula}

\begin{formula}
	{D = \left(\frac{n_2}{n_1} -1\right)\left(\frac{1}{r_1}+\frac{1}{r_2}\right)}
	D & Dioptrie & \\
	n & Brechungsindex & [] \\
	r_x & Radius & []
\end{formula}
\includegraphics[width=0.5\columnwidth]{Images/optiker}


\subsection{Fotoapperat}
In der Filmebene ergibt sich vom Punkt $G$ kein scharfer Bildpunkt, sondern ein Unscharferkreis mit dem Durchmesser $u$
\begin{formula}
	{\frac{1}{g} = \frac{1}{g_0}\pm\frac{u}{qf^2}}
	D & Dioptrie & \\
	f & Brennweite & [m]
\end{formula}
\begin{center}
	\includegraphics[width=\columnwidth]{Images/fotoapperat}
\end{center}

Die Bildweite $b$ ist normalerweise kleiner als die Gegenstandsweite $g$. Dabei ist die Intensität der Lichtstärke $H$ gegeben durch
\begin{formulaexpanded}
	{B = \frac{f}{g}G \qquad H = \left(\frac{d}{f}\right)^2 = q^2}
	Z = \frac{1}{q} & Blendenzahl & [1, 1.4, 2...] \\
	H & Lichtstärke & [] \\
	f & Brennweite & [] \\
	d &  & []
\end{formulaexpanded}

\subsection{Mikroskop}
\includegraphics[width=\columnwidth]{Images/mikroskop}
Die Vergrösserung ist gemäss Bild
\begin{formulaexpanded}
	{V = \frac{\tan\varepsilon}{\tan\varepsilon_0} = \frac{\frac{B}{f_2}}{\frac{G}{s}}= \frac{B}{G}\frac{s}{f_2}=\frac{b_1}{g_1}\frac{s}{f_2} = \frac{\Delta \cdot s}{f_1f_2}}
	s & Sehweite & [= ~25cm] 
\end{formulaexpanded}



\subsection{Schwingungen}
\todo{Beispiel mit Feder Prüfung 2019 Aufgabe 2}

\subsection{Pendel}
Differentialgleichung aufstellen für Pendel mit Gravitation. $l=87cm$ mit Masse von $930g$, am Ende ist eine quadratische Metallplatte mit Seitenlänge $10cm$ und einer Masse von $1250g$ befestigt. Bei kleiner Auslenkung gilt:
\begin{center}
	\includegraphics[width=0.6\columnwidth]{Images/pendel}
\end{center}
\textbf{Schwerepunkt} $a$
\begin{align*}
	a_y = \frac{m \cdot\frac{l}{2} + M\cdot l}{m+M} = 0.713m
\end{align*}
\textbf{Trägheitsmoment} $J$
\begin{align*}
	J_S &= \frac{1}{12}ml^2 + \overbrace{m\left(\frac{l}{2}\right)^2}^{\text{S.v.Steiner}} = \frac{1}{3}ml^2 = 0.235kgm^3 \\
	J_P &= \frac{1}{12}M(D^2 + D^2) + \underbrace{ML^2}_{\text{S.v.Steiner}} = 1.060kgm^3 \\
	J &= J_S + J_P
\end{align*}
Durch einsetzen in die Allgemeine Formel für Pendel $T(a) = 2\pi\sqrt{\frac{J}{mga}}$ ergibt sich $1.83s$


\subsection{Radar}
Wenn sich ein Objekt dem Sender mit Frequenz $f_s$ nähert, dann gilt der optische Doppler-Effekt doppelt. Das bedeutet, dass beim Objekt die reflektierte Frequenz \[f' = \frac{\sqrt{1 - \beta^2}}{1 - \beta}\cdot f_s\]wobei $\beta = \frac{u}{c}$, auf die Empfänger-Frequenz $f_e$ angewendet werden muss:
\begin{align*}
	f_e &=  \frac{\sqrt{1 - \beta^2}}{1 - \beta} \cdot \underbrace{\frac{\sqrt{1 - \beta^2}}{1 - \beta}\cdot f_s}_{f'} \\
	 &= \frac{1-\beta^2}{(1-\beta)^2}\cdot f_s \\
	 &= \frac{1 + \frac{u}{c}}{\frac{u}{c}}
\end{align*}
Siehe auch Schwebung im Kapitel \ref{schwebung}


\section{Anwendungen}
\subsection{Parabolspiegel}
Wobei $f$ die Brennweite ist. Damit lässt sich eine Funktion für den Parabolspiegel finden bei dem der Brennpunkt bestimmt ist.
\[y(x) = \frac{1}{4f}x^2  - f\]

\begin{center}
	\includegraphics[width=0.8\columnwidth]{Images/parabolspiegel}
\end{center}


\subsection{Spiegelungen}
Bei einem Spiegel kann das virtuelle Bild durch eine Spiegelung des Bildpunkts und dem Strahlensatz gefunden werden.
\includegraphics[width=\linewidth]{Images/spiegel_anwendung}


\subsubsection{Optiker}
\begin{formula}
	{D = \frac{1}{f}}
	D & Dioptrie & \\
	f & Brennweite & [m]
\end{formula}

\begin{formula}
	{D = \left(\frac{n_2}{n_1} -1\right)\left(\frac{1}{r_1}+\frac{1}{r_2}\right)}
	D & Dioptrie & \\
	n & Brechungsindex & [] \\
	r_x & Radius & []
\end{formula}
\includegraphics[width=0.5\columnwidth]{Images/optiker}


\subsection{Fotoapperat}
In der Filmebene ergibt sich vom Punkt $G$ kein scharfer Bildpunkt, sondern ein Unscharferkreis mit dem Durchmesser $u$
\begin{formula}
	{\frac{1}{g} = \frac{1}{g_0}\pm\frac{u}{qf^2}}
	D & Dioptrie & \\
	f & Brennweite & [m]
\end{formula}
\begin{center}
	\includegraphics[width=\columnwidth]{Images/fotoapperat}
\end{center}

Die Bildweite $b$ ist normalerweise kleiner als die Gegenstandsweite $g$. Dabei ist die Intensität der Lichtstärke $H$ gegeben durch
\begin{formulaexpanded}
	{B = \frac{f}{g}G \qquad H = \left(\frac{d}{f}\right)^2 = q^2}
	Z = \frac{1}{q} & Blendenzahl & [1, 1.4, 2...] \\
	H & Lichtstärke & [] \\
	f & Brennweite & [] \\
	d &  & []
\end{formulaexpanded}

\subsection{Mikroskop}
\includegraphics[width=\columnwidth]{Images/mikroskop}
Die Vergrösserung ist gemäss Bild
\begin{formulaexpanded}
	{V = \frac{\tan\varepsilon}{\tan\varepsilon_0} = \frac{\frac{B}{f_2}}{\frac{G}{s}}= \frac{B}{G}\frac{s}{f_2}=\frac{b_1}{g_1}\frac{s}{f_2} = \frac{\Delta \cdot s}{f_1f_2}}
	s & Sehweite & [= ~25cm] 
\end{formulaexpanded}


\subsection{Schwingungen}
Hantel mit $m=20kg$ besteht aus 2 Kugeln mit je $r_1 = 13.5cm$ Durchmesser. Die Verbindungsstange ist $l=25cm$ mit $m=5kg$. Die Hantel wird im Schwerpunkt aufgehängt und in Drehschwigung gesetzt. Schwingdauer ist $T_1 = 2s$. Welche Schwingdauer hat eine Hantel mit total $25kg$ und je $r2= 15.5cm$ Kugeln.
\begin{center}
	\includegraphics[width=0.4\columnwidth]{Images/schwingung}
\end{center}
\begin{align*}
	J_{S} &= \frac{1}{12}ml^2 = 0.026kgm^2 \\
	J_{K} &= \frac{2}{5}mR^2 \xrightarrow{} J_{K1} = 0.0137kgm^2, J_{K2} = 0.024kgm^2 \\
	J_{Hn} &= J_{S} + 2\left(J_{Kn} + m_{Kn}\left(\frac{l}{2}+r_{Kn}\right)^2\right) \\
	& \xrightarrow{} J_{H1} = 0.609kgm^2, J_{H2} = 0.894kgm^2
\end{align*}
Dadurch lässt sich ein Verhältniss der zwei Experimente bilden
\begin{align*}
	T &= 2\pi \sqrt{\frac{J}{c}} \xrightarrow{} \frac{T_1}{T_2} = \sqrt{\frac{J_1}{J_2}}\\
	\xRightarrow[]{} T_2 &= T_1 \cdot \frac{J_2}{J_1} = 2.42s
\end{align*}

\subsection{Pendel}
Differentialgleichung aufstellen für Pendel mit Gravitation. $l=87cm$ mit Masse von $930g$, am Ende ist eine quadratische Metallplatte mit Seitenlänge $10cm$ und einer Masse von $1250g$ befestigt. Bei kleiner Auslenkung gilt:
\begin{center}
	\includegraphics[width=0.6\columnwidth]{Images/pendel}
\end{center}
\textbf{Schwerepunkt} $a$
\begin{align*}
	a_y = \frac{m \cdot\frac{l}{2} + M\cdot l}{m+M} = 0.713m
\end{align*}
\textbf{Trägheitsmoment} $J$
\begin{align*}
	J_S &= \frac{1}{12}ml^2 + \overbrace{m\left(\frac{l}{2}\right)^2}^{\text{S.v.Steiner}} = \frac{1}{3}ml^2 = 0.235kgm^3 \\
	J_P &= \frac{1}{12}M(D^2 + D^2) + \underbrace{ML^2}_{\text{S.v.Steiner}} = 1.060kgm^3 \\
	J &= J_S + J_P
\end{align*}
Durch einsetzen in die Allgemeine Formel für Pendel $T(a) = 2\pi\sqrt{\frac{J}{mga}}$ ergibt sich $1.83s$


\subsection{Radar}
Wenn sich ein Objekt dem Sender mit Frequenz $f_s$ nähert, dann gilt der optische Doppler-Effekt doppelt. Das bedeutet, dass beim Objekt die reflektierte Frequenz \[f' = \frac{\sqrt{1 - \beta^2}}{1 - \beta}\cdot f_s\]wobei $\beta = \frac{u}{c}$, auf die Empfänger-Frequenz $f_e$ angewendet werden muss:
\begin{align*}
	f_e &=  \frac{\sqrt{1 - \beta^2}}{1 - \beta} \cdot \underbrace{\frac{\sqrt{1 - \beta^2}}{1 - \beta}\cdot f_s}_{f'} \\
	 &= \frac{1-\beta^2}{(1-\beta)^2}\cdot f_s \\
	 &= \frac{1 + \frac{u}{c}}{1- \frac{u}{c}}
\end{align*}
Siehe auch Schwebung im Kapitel \ref{schwebung}

\subsection{Interferenz}
Zwei Lautsprecher sind im Abstand von $s=80cm$ aufgestellt und spielen einen Sinus Ton unterhalb von $500Hz$ ab. Im Abstand von $R=4.8m$ mit Mittelpunkt zwischen den Lautsprechner, wird der Ton bei $30^\circ$ zur Verbindungslinie am leisesten. Temperatur ist $20^\circ C$.
\begin{center}
	\includegraphics[width=0.6\columnwidth]{Images/interferenz}
\end{center}
\begin{align*}
	u = \sqrt{\frac{\chi R T}{M_{Luft}}} = \sqrt{\frac{1.4\cdot 8.314 \cdot (273 + 20)}{0.0288}} = 344\frac{m}{s}
\end{align*}
Ein Auslöschung findet bei $\Delta r = r_2 - r_1 = \frac{\lambda}{2}(2n + 1)$ mit $n \in \mathbb{N^+}$ statt. Mittels Cosinus-Satz (Achtung 2 Lösungen!) und $\lambda = \frac{u}{f}$ kann folgende Formel gefunden werden
\begin{align*}
	r_1^2 &= R^2 + \left(\frac{s}{2}\right)^2 - 2R\frac{s}{2}\cos(30) \quad r_2^2 = R^2 + \left(\frac{s}{2}\right)^2 + 2R\frac{s}{2}\cos(150) 	\\
	\Delta r &= r_2 - r_1 = 0.692m \\
	\Delta r &= \frac{\lambda}{2}(2n + 1) \xRightarrow[]{\lambda = \frac{u}{f}}f = \frac{u}{2\Delta r}(2n +1) \\
	&\xrightarrow{} n=0: f_1 = 249Hz, \quad n=1: f_2 = 746Hz (\text{keine Lösung})
\end{align*}

\subsection{Unwucht}
Ein Pneu mit $m=22kg$ und $d = 15 Zoll = 15\cdot0.0254 = 0.381m$ wird ein Bleigewicht von $30g$ befestigt, um die Unwucht zu korrigieren. $e$ ist der Abstand zwischen Drehachse und Schwerpunkt (Exzentrität)
\begin{align*}
	M_{Blei} = F_{Blei}\cdot \frac{d}{2} =  9.81 \cdot 0.03 \cdot 0.19 = 0.056Nm \\
	e = \frac{M_{Blei}}{M_{Rad}} = 0.28mm
\end{align*}
Bei $72km/h$ ergibt sich dadurch eine Radialkraft von $F_{rad} = a_r\cdot m = \omega^2e\cdot m$ wobei $\omega = \frac{v}{d/2} = 63rad/s$ ist.
\section{Schwingung}
Wird ein System aus dem Gleichgewicht gebracht und dann sich selbst überlassen, so führt dies zu einer \textbf{freien Schwingung}. Wird das System von aussen durch Störungen zum Schwingen erregt, so wird von \textbf{erzwungenen} Schwigung gesprochen. Ein schwingfähiges System kann unter Umständen auch eine Energiequelle sein und \textbf{selbsterregte Schwingung} erzeugen.

\noindent Eine reine harmonische Kinematik-Schwingung ist definiert mit $\omega = 2\pi f$ durch
\[
y(t) = A\cdot \sin(\omega t + \varphi)
\]


\subsection{Eigenfrequenz}
\begin{formula}
	{\omega_0 = \sqrt{\frac{c}{m}} \qquad D = \frac{\delta}{\omega_0}}
	\omega_0 & Eigenfrequenz & \\
	\delta & Ablingkonstante &
\end{formula}

\noindent Die \textbf{Ablingkonstante} kann bestimmt werden durch Messen einer Schwingung $f(x) = \sin(2\pi f x)\cdot e^{-\delta x}$ von zwei aufeinanderfolgenden Amplituden $\max A, B$ und der Periode $T$. Wobei \[\Lambda = \underbrace{\ln\left(\frac{y(A)}{y(B)}\right)}_\delta \cdot \frac{1}{T}\]
\begin{center}
	\includegraphics[width=0.8\columnwidth]{Images/ablingkonstante}
\end{center}
	
\begin{formula}
	{\omega_d = \sqrt{\omega_0^2 - \delta^2}}
	\omega_d & Gedämpfte Schwingung & \\
	D  & Dämpfungsgrad & 
\end{formula}

\begin{formula}
	{\omega_r = \sqrt{\omega_0^2 - 2\delta^2}}
	\omega_r & Erzwungene Schwingung & \\
	D  & Dämpfungsgrad & 
\end{formula}
	
\subsection{DGL mit konstanten Koeffizienten}\label{DGL}
DGL mit konstanten Koeffizieten können durch Schema F gelöst werden:
\[
\textcolor{red}{a} \cdot \ddot{x}(t) + \textcolor{blue}{b} \cdot \dot{x}(t) = 0 \quad \xRightarrow[\quad]{}  \quad  \ddot{x}(t) + \underbrace{\frac{\textcolor{blue}{b}}{\textcolor{red}{a}}}_{\omega_0^2}\cdot \dot{x}(t) = 0
\]
Die Lösung der DGL ist $x(t) = A \cdot \left(\omega_0 \cdot t + \varphi_0\right)$
\[
\omega_0 = \sqrt{\frac{\textcolor{blue}{b}}{\textcolor{red}{a}}} \qquad T = \frac{2\pi}{\omega_0} = 2\pi\sqrt{\frac{\textcolor{red}{a}}{\textcolor{blue}{b}}}
\]


\subsection{Trägheitsmoment}
\includegraphics[width=\columnwidth]{Images/j}

\noindent Mittels dem Satz von Steiner kann auch das Trägheitsmoment um eine Drehachse, welche nicht die Schwerachse ist, berechnet werden. Sie Achsen müssen jedoch parallel sein! \kuchling{131}
\begin{formula}
	{J_A = J_S + md^2}
	J_A & Drehmoment um Achse & \\
	J_S & Drehmoment um Schwereachse & [] \\
	d & Abstand von Drehachse zur Schwereachse & [] \\
	m & Masse & []
\end{formula}

\subsection{Schwebung}
\todo{}


